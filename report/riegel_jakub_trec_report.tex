  % !TeX spellcheck = en_US
\documentclass[10pt, a4paper, twocolumn]{article}

\usepackage[utf8]{inputenc}
\usepackage{graphicx}
\usepackage[margin=2.5cm]{geometry}
\usepackage{csquotes}
\usepackage{listings, xcolor}
\usepackage[hidelinks]{hyperref}
\usepackage{lipsum}

\title{ZTI 2020\newline TREC Project }
\date{\today }
\author{Jakub Riegel \\ jakub.riegel@student.put.poznan.pl }

\lstset{
	basicstyle=\small\ttfamily,
	columns=flexible,
	breaklines=true
}

\begin{document}
	\maketitle
	
	\begin{abstract}
		\lipsum[2]
	\end{abstract}
	
	\section{Introduction}
		As rapid technical advance of recent decades is still accelerating, the new era of information technology emerges. Artificial Intelligence (AI) has a great chance to change human life in almost every aspect. Today's world is overwhelmingly complicated mechanism and no different is it with AI. Starting with the question \emph{What is Artificial Intelligence?} the answer is a little bit ambiguous. For instance one encyclopedia says that:
		
		\begin{displayquote}
			Artificial Intelligence (AI) - a field of science dealing with the study of the mechanisms of human intelligence  and modeling and construction of systems that are able to support or replace intelligent human actions.\footnote{Encyklopedia PWN accessed in December 2020 at \href{https://encyklopedia.pwn.pl/szukaj/sztuczna\%20inteligencja.html}{encyklopedia.pwn.pl/szukaj/sztuczna\%20inteligencja.html}.}
		\end{displayquote}

		Whereas a subject-oriented publication \cite{newsynthesis} states that \emph{(AI) involves perception, reasoning, learning, communicating and acting in complex environments}. But few years later \cite{thequest} the same author says \emph{Artificial intelligence is that activity devoted to making machines intelligent, and intelligence is that quality that enables an entity to function appropriately and with foresight in its environment}. Variance in defining this term may lead to conclusion, that what AI really is depends on responsibility one has given to one's implementation\cite{ai100}. More specific disambiguation wheather given application is AI on not lies beyond the scope of this paper.
		
		Some of key areas of AI implementation are machine learning (ML), natural language processing (NLP), expert systems, computer vision and general automation. A note has to be given on application of this technologies. Since technology in general is made for improving people's life, most spectacular TBATBATBA will be visible in areas other than informatics, like medicine, space engineering, economics or transport. 
		
	\section{Coclusion}
		\lipsum[4]
	
	\section*{Acknowledgements}
	This paper was made as a complimentary for TUM admission process.
	
	\bibliography{bib} 
	\bibliographystyle{ieeetr}
\end{document}